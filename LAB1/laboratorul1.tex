\documentclass[12pt,a4paper,titlepage]{article}
\usepackage {graphicx}
 
\begin{document}

\begin{titlepage}
\selectlanguage{english}

%----------------------------------------------------------------------------------------
% TITLE PAGE INFORMATION
%----------------------------------------------------------------------------------------
  \begin{center} % Center everything on the page

  %----------------------------------------------------------------------------------------
  % HEADING SECTIONS
  %----------------------------------------------------------------------------------------
  \textsc{\large Facultatea Calculatoare, Informatica si Microelectronica}\\[0.5cm]
  \textsc{\large Universitatea Tehnica a Moldovei}\\[1.2cm] % Name of your university/college
  \vspace{25 mm}

  \textsc{\Large Medii Interactive de Dezvoltare a Produselor Soft}\\[0.5cm] % Major heading such as course name
  \textsc{\large Lucrarea de laborator\#1}\\[0.5cm] % Minor heading such as course title
  %\textsc{\large Laboratory work}\\[0.5cm] % Minor heading such as course title

\newcommand{\HRule}{\rule{\linewidth}{0.5mm}} % Defines a new command for the horizontal lines, change thickness here

  %----------------------------------------------------------------------------------------
  % TITLE SECTION
  %----------------------------------------------------------------------------------------
  \vspace{10 mm}
  \HRule \\[0.4cm]
  { \LARGE \bfseries Version Control Systems si modul de setare a unui server }\\[0.4cm] % Title of your document
  \HRule \\[1.5cm]

  %----------------------------------------------------------------------------------------
  % AUTHOR SECTION
  %----------------------------------------------------------------------------------------
      \vspace{30mm}

      \begin{minipage}{0.4\textwidth}
      \begin{flushleft} \large
      \emph{Autor:}\\
      Toma {Ana}
      \end{flushleft}
      \end{minipage}
      ~
      \begin{minipage}{0.4\textwidth}
      \begin{flushright} \large
      \emph{lector asistent:} \\
      Irina {Cojanu} \\ 
      \emph{lector superior:} \\
      Radu {Melnic} % Supervisor's Name
      \end{flushright}
      \end{minipage}\\[4cm]

      \vspace{5 mm}
      % If you don't want a supervisor, uncomment the two lines below and remove the section above
      %\Large \emph{Author:}\\
      %John \textsc{Smith}\\[3cm] % Your name

      %----------------------------------------------------------------------------------------
      % DATE SECTION
      %----------------------------------------------------------------------------------------

      %{\large \today}\\[3cm] % Date, change the \today to a set date if you want to be precise

      %----------------------------------------------------------------------------------------
      % LOGO SECTION
      %----------------------------------------------------------------------------------------

      %\includegraphics{red}\\[0.5cm] % Include a department/university logo - this will require the graphicx package

      %----------------------------------------------------------------------------------------

      \vfill % Fill the rest of the page with whitespace
      \end{center}
      
\end{titlepage}
   
 \section{Scopul lucrarii de laborator}
De a se invata utilizarea unui Version Control System si modul de setare a unui server.
\section{Obiective}
Studierea Version Control Systems (git).
\clearpage
\section{Mersul lucrarii de laborator}

\subsection{Cerintele}

1.Initializarea unui nou repositoriu.\\
2.Configurarea VCS.\\
3.Commit, Push pe branch.\\
4.Folosirea fisierului .gitignore.\\
5.Revenire la versiunele anterioare.\\
6.Crearea branch-urilor noi.\\
7.Commit pe ambele branch-uri.\\
8.Merge la 2 branchuri.\\
9.Rezolvarea conflictelor.\\

\subsection{Analiza Lucrarii de laborator}
\tab Linkul la repozitoriu \textbf{https://github.com/TomaAna/MIDPS}\\
Sunt mai multe modalitati de a initializa un repozitoriu pe github.
Putem crea o mapa goala in care vom plasa gitul nostru prin intermediul comenzii \textbf{git init}.\\
\tab Urmatorul pas este crearea insusi a noului repozitoriu pe care il vom crea utilizind urmatoarea comanda
\textbf{curl -u 'USER' https.//api.github.com\\/user/repos -d '\{"name":"NUME"\}'}. Unde cuvintele scrise cu CAPS
se vor inlocui cu numele utilizatorului si numele repozitoriului.\\
\tab Dupa aceasta este necesar sa unim gitul nostru gol cu repozitoriul creat.
Vom folosi urmatoarea comanda \textbf{git remote add origin "Linkul la repositoriu"}\\

\tab O alta metoda de a crea un repozitoriu este cea online.
Pentru aceasta este nevoie sa deschidem pagina noastra pe github , sa alegem \textbf{repositories} si sa apasam butonul \textbf{new.}\\
\\

\\
\\
\tab \textbf{Configurarea gitului} consta in mai multe etape. La inceput vom configura numele si emailul.
Scrim urmatoarele comenzi:\\
\textbf{git config --global user.name "NUMELE"}\\
\textbf{git config --global user.email EMAIL}\\
\includegraphics [width=\textwidth]{lion.png}

\\

\tab Urmatorul pas consta in generarea la cheia \textbf{SSH} (Secure Shell). Scriem in CLI \textbf{ssh-keygen},
iar cheia obtinuta o copiem in setarile noastre de pe git.
\tab Este de dorit sa initializam repozitoriul nostru cu un fisier \textbf{README.md} si un \textbf{.gitignore}. In fisierul
README.md vom adauga niste informatie pentru cei care se vor folosi de repozitoriu iar in fisierul .gitignore vom adauga
toate fisierele ce trebuiesc ignorate (adica sa nu fie incarcate).\\
\includegraphics[width=\textwidth]{ssh.png}
\clearpage

\textbf Dupa ce am generat keygen-ul,clonam repozitoriul pe masina locala.
\includegraphics[width=\textwidth]{masina.png}
\textbf Pentru a adauga fisiere pe repozitoriu,vom folosi urmatoarele comenzi: git add * - comanda
indexeaza toate fisierele. git commit -m - comanda face un snapshot la toate schimbarile
noastre.
git push origin master - comanda incarca toate fisierele indexate pe git.Totodata vom
folosi git status si git show pentru a ne asigura ca fisierele au fost adaugate in repozitoriu.\textbf\\
\includegraphics[width=\textwidth]{versiune1.png}
\clearpage
\textbf Revenirea la o versiune mai veche poate fi efectuata cu ajutorul comenzii git reset �TYPE
�codul comitului�. Exista diferenta intre �soft si �hard , cind facem soft reset indexurile
ramin neschimbate. Iar in cazul in care facem hard reset , pierdem indexurile.
Am creat un fisier nou text.txt in versiunea 1.\tab Dupa care l-am sters si am facut commit la
versiunea 2 in care am sters fisierul test.txt.Dorim sa revenim la versiunea1.\textbf La inceput
vom lansa comanda git log care ne arata logul de commituri si codul pentru fiecare
commit. Vom avea nevoie de primele 7 cifre la commitul anterior.\textbf\\

\includegraphics[width=\textwidth]{gitlog.png}\\
\textbf Acum folosim comenzile git reset --hard si git reset --soft.\\ \includegraphics[width=\textwidth]{gitreset.png}\\
VCS ne permite sa avem mai multe branch-uri. Branch-urile sunt comod de folosit cind
dorim sa lucram paralel la un proiect si apoi dorim sa unim toate modificarile.
git branch �name� - creeaza un branch nou cu numele �name�. git branch - vizualizarea
branch-urilor (* indica branch-ul curent). git branch -d �nume� - sterge branch-ul
�nume�. git checkout -b �name� - creeaza un branch nou cu numele �name� si face switch
la el.\\
\includegraphics[width=\textwidth]{branch.png}
\includegraphics[width=\textwidth]{branchnou.png}
\includegraphics[width=\textwidth]{brr.png}
\clearpage
Vom lucra cu 2 branch-uri - �master� si �nou�. Vom crea in fiecare branch cite un fisier
to mer,dar continutul fiecaruia va fi diferit.\\
\includegraphics[width=\textwidth]{conflict.png}
\clearpage
\section{Concluzie}
 \textbf  In urma efectuarii lucrarii de laborator numarul 1 la MIDP am studiat\textbf { VCS}.Mi-am aprofundat cunostintele in folosirea platformei GitHub.
Am analizat initializarea unui nou repositoriu si am invatat cum sa creez branch-uri noi.
A fost o experienta noua pentru mine,deoarece anterior nu am folosit git-ul.
Astfel am aflat ca git-ul este un sistem open-source de control a versiunilor conceput de Linus Trovalds � aceea?i persoana care a creat sistemul de operare Linux. Git este in fapt asemanator cu alte astfel de sisteme de control a versiunilor, precum Mercurial, Subversion sau CVS.
Git este efectiv o unealta bazata pe linii de comanda, insa locul in care se centralizeaza toate datele si in care are loc stocarea proiectelor este efectiv Hub-ul, mai exact GitHub.com. Aici dezvoltatorii pot adauga si stoca proiecte la care lucreaza impreuna cu alti pasionati.
\end{document}



