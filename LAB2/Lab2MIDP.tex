\documentclass[12pt,a4paper,titlepage]{article}
\usepackage {graphicx}
 
\begin{document}

\begin{titlepage}
\selectlanguage{english}

%----------------------------------------------------------------------------------------
% TITLE PAGE INFORMATION
%----------------------------------------------------------------------------------------
  \begin{center} % Center everything on the page

  %----------------------------------------------------------------------------------------
  % HEADING SECTIONS
  %----------------------------------------------------------------------------------------
  \textsc{\large Facultatea Calculatoare, Informatica si Microelectronica}\\[0.5cm]
  \textsc{\large Universitatea Tehnica a Moldovei}\\[1.2cm] % Name of your university/college
  \vspace{25 mm}

  \textsc{\Large Medii Interactive de Dezvoltare a Produselor Soft}\\[0.5cm] % Major heading such as course name
  \textsc{\large Lucrarea de laborator\#2}\\[0.5cm] % Minor heading such as course title
  %\textsc{\large Laboratory work}\\[0.5cm] % Minor heading such as course title

\newcommand{\HRule}{\rule{\linewidth}{0.5mm}} % Defines a new command for the horizontal lines, change thickness here

  %----------------------------------------------------------------------------------------
  % TITLE SECTION
  %----------------------------------------------------------------------------------------
  \vspace{10 mm}
  \HRule \\[0.4cm]
  { \LARGE \bfseries GUI Development }\\[0.4cm] % Title of your document
  \HRule \\[1.5cm]

  %----------------------------------------------------------------------------------------
  % AUTHOR SECTION
  %----------------------------------------------------------------------------------------
      \vspace{30mm}

      \begin{minipage}{0.4\textwidth}
      \begin{flushleft} \large
      \emph{Autor:}\\
      Toma {Ana}
      \end{flushleft}
      \end{minipage}
      ~
      \begin{minipage}{0.4\textwidth}
      \begin{flushright} \large
      \emph{lector asistent:} \\
      Irina {Cojanu} \\ 
      \emph{lector superior:} \\
      Radu {Melnic} % Supervisor's Name
      \end{flushright}
      \end{minipage}\\[4cm]

      \vspace{5 mm}
      % If you don't want a supervisor, uncomment the two lines below and remove the section above
      %\Large \emph{Author:}\\
      %John \textsc{Smith}\\[3cm] % Your name

      %----------------------------------------------------------------------------------------
      % DATE SECTION
      %----------------------------------------------------------------------------------------

      %{\large \today}\\[3cm] % Date, change the \today to a set date if you want to be precise

      %----------------------------------------------------------------------------------------
      % LOGO SECTION
      %----------------------------------------------------------------------------------------

      %\includegraphics{red}\\[0.5cm] % Include a department/university logo - this will require the graphicx package

      %----------------------------------------------------------------------------------------

      \vfill % Fill the rest of the page with whitespace
      \end{center}
      
\end{titlepage}
   
 \section{Scopul lucrarii de laborator}
Realizarea unui simplu GUI Calculator

\section{Obiective}
1.Realizeaza un simplu GUI Calculator\\
2.Operatiile simple: +,-,*,/,putere,radical,InversareSemn(+/-),operatii cu numere zecimale.\\
3.Divizare proiectului in doua module - Interfata grafica(Modul GUI) si Modulul de baza(Core Module).
\clearpage
\section{Mersul lucrarii de laborator}

\subsection{Cerintele}

1.Realizeaza un simplu GUI calculator care suporta urmatoare functii: +, -, /, *, putere, radical, InversareSemn(+/-), operatii cu numere zecimale.\\
2.Divizare proiectului in doua module - Interfata grafica(Modul GUI) si Modulul de baza(Core Module).\\

\subsection{Analiza Lucrarii de laborator}
\tab Linkul la repozitoriu \textbf{https://github.com/TomaAna/MIDPS}\\
Sunt mai multe modalitati de  realizare a unui gui calculator.
Pentru creaarea calculatorului am folosit programarea in windows cu ajutorul limbajului de programare  \textbf{C++}.\\
\tab Primul pas este crearea functiei \textbf {WINMAIN} care este eschivalentul WINDOWS a functiei main utilizata in toate programele scrise
in C si C++,folosita pentru prelucrari primare.Functia WinMain difera insa in multe privinte de main si nu in ultimul rind,prin modul de
de declarare.
\textbf{int WINAPI WinMain(HINSTANCE hInst,HINSTANCE hPrev,LPSTR CmdLine,int CmdShow)}. 

 \tab Functia WinMain returneaza  o valoare int la fel ca multe alte programe in C++.
 Parametrii acceptati de WinMain: \\ \textbf{WNDCLASS Wc;}\\
\textbf{ MSG      Msg;}\\

\tab Functia WinMain din program termina cu o bucla \textbf{while} care preia mesaje pina cind utilizatorul trimite sistemului mesajul 
     \textbf{WMQUIT}                 
\\

\\
\\
\tab \textbf{Bucla While}\\ 
while(GetMessage(&Msg,NULL,0,0))\\
   {\\
    HWND hActiveWindow = GetActiveWindow();\\
    if(!IsWindow(hActiveWindow) || !IsDialogMessage(hActiveWindow,&Msg))\\
      {\\
        TranslateMessage(&Msg);\\
        DispatchMessage(&Msg);\\
      }\\
    }\\
 return Msg.wParam;\\
}\\
\\

\tab Programul folosete clasa \textbf{BUTTON} pentru a crea butoane in cadrul ferestrelor.\\ 
     
Butoanele sunt butoane,de apasare sunt in forma de dreptunghi.\\
     
HWND BCX_Butoane \\
(char* Text,HWND hWnd,int id,int X,int Y,int W,int H,int Style,int Exstyle)\\
{
  HWND  A;\\
  
  if(!Style)\\
  {
    Style=WS_CHILD | WS_VISIBLE | BS_MULTILINE | BS_PUSHBUTTON | WS_TABSTOP;\\
  }
  if(Exstyle==-1)\\
  {
    Exstyle=WS_EX_STATICEDGE;\\
  }
  A = CreateWindowEx(Exstyle,"button",Text,Style,\\
    X*BCX_ScaleX, Y*BCX_ScaleY, W*BCX_ScaleX, H*BCX_ScaleY,\\
    hWnd,(HMENU)id,BCX_hInstance,NULL);\\
  SendMessage(A,(UINT)WM_SETFONT,(WPARAM)GetStockObject(DEFAULT_GUI_FONT),\\
    (LPARAM)MAKELPARAM(FALSE,0));\\
  if (W==0)\\
  {\\
    HDC  hdc=GetDC(A);\\
    SIZE  size;\\
    GetTextExtentPoint32(hdc,Text,strlen(Text),&size);\\
    ReleaseDC(A,hdc);\\
    MoveWindow(A,X*BCX_ScaleX,Y*BCX_ScaleY,(int)(size.cx+(size.cx*1)),(int)(size.cy+(size.cy*0.32)),TRUE);\\
  }\\
  return A;\\
}\\

HWND BCX_Editeaza\\
(char* Text,HWND hWnd,int id,int X,int Y,int W,int H,int Style,int Exstyle)\\
{
  HWND  A;\\
  // assign default style\\
  if (!Style)\\
  {
      Style = WS_CHILD | WS_VISIBLE | ES_WANTRETURN |\\
              WS_VSCROLL | ES_MULTILINE | ES_AUTOVSCROLL | ES_AUTOHSCROLL;\\
  }\\
  if (Exstyle==-1)\\
  {
      Exstyle = WS_EX_CLIENTEDGE;\\
  }\\
  A = CreateWindowEx(Exstyle,"edit",Text, Style,\\
      X*BCX_ScaleX, Y*BCX_ScaleY, W*BCX_ScaleX, H*BCX_ScaleY,\\
      hWnd,(HMENU)id,BCX_hInstance,NULL);\\

  SendMessage(A,(UINT)WM_SETFONT,(WPARAM)GetStockObject(DEFAULT_GUI_FONT),\\
    (LPARAM)MAKELPARAM(FALSE,0));\\
  return A;\\
}
char *BCX_Get_Text(HWND hWnd)\\
{\\
  int tmpint;\\
  tmpint = 2 + GetWindowTextLength(hWnd);\\
  char *strtmp = BCX_TmpStr(tmpint);\\
  GetWindowText(hWnd,strtmp,tmpint);\\
  return strtmp;\\
}\\
int BCX_Set_Text(HWND hWnd, char *Text)\\
{\\
  return SetWindowText(hWnd,Text);\\
}\\



 
\section{Concluzie}
 \tab  In urma efectuarii lucrarii de laborator numarul 2 la MIDPS am studiat  si am invatat cum sa realizez un simplu GUI calculator care
 suporta urmatoare functii: +, -, /, *, putere, radical, InversareSemn(+/-), operatii cu numere zecimale.
 Am folosit un limbaj cu care am facut cunostinta recent,astfel am invatat si am analizat mai multe lucruri noi.
 Am utilizat  un IDE pentru limbajele de programare C++, C ce a fost lansat in versiune stabila in 2008 care poarta denumirea de
\textbf{Code::Blocks} care permite  proiectarea interfetelor grafice intr-un mod vizual, de tipul WYSIWYG (What You See Is What You Get).
Designerul se numeste wxSmith si este derivat din libraria wxWidgets, librarie ce permite crearea de interfete grafice cross-platform.
      
           
     
\end{document}



